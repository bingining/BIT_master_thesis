% !Mode:: "TeX:UTF-8" 
% 以上一行不可删除,是为了确保WinEdit能够以UTF-8的格式打开本文件。

%%%%%%%%%%%%%%%%%%%%%%%%%%%%%%%%%%%%%%%%%%%%%%%%%%%%%%%%%%%%%%%%%%%%%%
% Read me %
%%%%%%%%%%%%%%%%%%%%%%%%%%%%%%%%%%%%%%%%%%%%%%%%%%%%%%%%%%%%%%%%%%%%%%
% 这是北京理工大学硕士研究生毕业论文的模板。
% 参考资料:北京理工大学博士、硕士学位论文撰写规范
% 具体规范见:http://www.bitme.com.cn/peiyang/2009-11-13/170.html
%
% 本模板在以下环境已测试可以编译通过:
% 操作系统:linux(ubuntu和archlinux)
% 文本编辑器:Texmaker
% LaTex编译选项:XeLaTex(必须用XeTex)
%
% 本模板最初由大眼小蚂蚁根据交大的模板修改得到,参见:
% https://code.google.com/p/bit-master-thesis-latex/
%
% 此版本由bingining做了大幅度改动,使得与学校所给的Word模板更一致。
%%%%%%%%%%%%%%%%%%%%%%%%%%%%%%%%%%%%%%%%%%%%%%%%%%%%%%%%%%%%%%%%%%%%%%

%%%%%%%%%%%%%%%%%%%%%%%%%%%%%%%%%%%%%%%%%%%%%%%%%%%%%%%%%%%%%%%%%%%%%%
% 导言 %
%%%%%%%%%%%%%%%%%%%%%%%%%%%%%%%%%%%%%%%%%%%%%%%%%%%%%%%%%%%%%%%%%%%%%%
\documentclass[twoside]{BitMasterThesis}
% twoside对应双面排版;如需单面排版,请用oneside选项;
% 请尽量使用twoside选项,因为学校的规范里面明确要求应该双面打印:
%	"论文稿件要求使用A4纸双面打印,其中摘要之前部分使用单页打印。"
% 另外oneside选项没经过调试,可能会存在问题。
% 其余的设置在BitMasterThesis.cls里头,已使用学校规定的设置,无需更改。


%%%%%%%%%%%%%%%%%%%%%%%%%%%%%%%%%%%%%%%%%%%%%%%%%%%%%%%%%%%%%%%%%%%%%%
% 宏包调用 %
% 原来的宏包调用都是写在bitmaster-xetex.cls里的。
% 经过修改后,% 除了一些必须的且无需更改的宏包仍留在bitmaster-xetex.cfg里面
% (包括:natbib,hyperref,geometry,setspace,titletoc,ifthen,
%		fancyhdr,fontspec,xltxtra,fontenc,mathptmx,ccaption)之外,
%	其余的宏包调用都需要在这个主文档里写明(例如处理公式、表格和图标等的宏包)。
%%%%%%%%%%%%%%%%%%%%%%%%%%%%%%%%%%%%%%%%%%%%%%%%%%%%%%%%%%%%%%%%%%%%%%
\usepackage{amsmath,amssymb}% 数学公式和字体 %
\usepackage{tensor}% 处理张量 %
\numberwithin{equation}{section}% 以章节来编号公式 %
\allowdisplaybreaks% 使得align环境公式可跨页 %
\usepackage{bm}% \bm{<text>} Bold math symbols %
%\usepackage{dcolumn}% Align table columns on decimal point  %
%\usepackage{array}% dcolumn depends on array %
\usepackage{graphicx}% 插入图片 %
\usepackage{ulem}% underlining for emphasis %
\usepackage{cancel}% Place lines through maths formulae %
%\usepackage{setspace}% 保留标题行距,改变正文行距 %


%%%%%%%%%%%%%%%%%%%%%%%%%%%%%%%%%%%%%%%%%%%%%%%%%%%%%%%%%%%%%%%%%%%%%%
% newcommand,尽量不要用renewcommand % 
%%%%%%%%%%%%%%%%%%%%%%%%%%%%%%%%%%%%%%%%%%%%%%%%%%%%%%%%%%%%%%%%%%%%%%
\newcommand*{\be}{\begin{equation}}
\newcommand*{\ee}{\end{equation}}
\newcommand*{\bea}{\begin{eqnarray}}
\newcommand*{\eea}{\end{eqnarray}}
\newcommand*{\p}{\partial}
\newcommand*{\om}{\omega}
\newcommand*{\Om}{\Omega}
\newcommand*{\mc}{\mathcal}
\newcommand*{\s}{\sigma}
\newcommand*{\ts}{\tensor}
\newcommand*{\g}{\gamma}
\newcommand*{\al}{\alpha}
\newcommand*{\bt}{\beta}
\newcommand*{\Ga}{\Gamma}
\newcommand*{\la}{\lambda}
\newcommand*{\dt}{\delta}
\renewcommand*{\cal}{\mathcal}



\begin{document}

%%%%%%%%%%%%%%%%%%%%%%%%%%%%%%%%%%%%%%%%%%%%%%%%%%%%%%%%%%%%%%%%%%%%%%
% 封面信息,需要修改,依照每个人的情况填写 %
%%%%%%%%%%%%%%%%%%%%%%%%%%%%%%%%%%%%%%%%%%%%%%%%%%%%%%%%%%%%%%%%%%%%%%
\classification{O412.1}% 中图分类号 %
% O412.1:相对论;见 http://www.ztflh.com/

\UDC{530}% UDC 分类号 %
% 531.5:Gravity. Gravitation. Pendulums. Ballistics
% 见 http://www.udcc.org/udcsummary

\title{\LARGE{基于绝对平行时空的修改引力理论的一些研究}}% 标题 %
\author{陈祖成}% 作者姓名 %
\institute{物理学院}% 学院名称 %
\advisor{韦浩教授}% 指导老师 %
\chairman{未定教授}% 答辩委员会主席 %
\degree{理学硕士}% 申请学位 %
\major{理论物理}% 学科专业 %
\school{北京理工大学}% 学位授予单位 %
\defenddate{2015年1月}% 论文答辩日期 %

\englishtitle{\LARGE{Some research of modified gravity theories
based on teleparallel space-time}}% english title %
\englishauthor{Zu-Cheng Chen}% Candidate Name %
\englishinstitute{School of Physics}% School or Department %
\englishadvisor{Prof. Hao Wei}% Faculty Mentor %
\englishchairman{Prof. Unknown}% Chair, Thesis Committee
\englishdegree{Master of Science}% Degree Applied %
\englishmajor{Theoretical Physics}% Major %
\englishschool{Beijing Institute of Technology}% Degree by %
\englishdate{January, 2015}% The Date of Defence %

%%%%%%%%%%%%%%%%%%%%%%%%%%%%%%%%%%%%%%%%%%%%%%%%%%%%%%%%%%%%%%%%%%%%%%
% 生成摘要之前的页面 %
%%%%%%%%%%%%%%%%%%%%%%%%%%%%%%%%%%%%%%%%%%%%%%%%%%%%%%%%%%%%%%%%%%%%%%

\maketitle% 生成中文封面及中文标题页 %

\makeenglishtitle% 生成英文标题页 %

\makeVerticalTitle% 打印竖排论文题目 %

\makeDeclareOriginal% 研究成果声明和关于学位论文使用权的说明


%%%%%%%%%%%%%%%%%%%%%%%%%%%%%%%%%%%%%%%%%%%%%%%%%%%%%%%%%%%%%%%%%%%%%%
% 设置页眉和页脚 % 
%%%%%%%%%%%%%%%%%%%%%%%%%%%%%%%%%%%%%%%%%%%%%%%%%%%%%%%%%%%%%%%%%%%%%%
\pagestyle{fancy}
\fancyhf{}
\lhead{}
\chead{\songti \zihao{5} 北京理工大学硕士学位论文}
\rhead{}
% 页眉: 宋体, 5号居中, “北京理工大学硕士学位论文”
\lfoot{}
\cfoot{\songti \zihao{5} \thepage}
\rfoot{}
% 页码:宋体, 5号页面底端居中

% 在以下定义plain格式,是为了让目录首页及各章节的起始页也显示页眉 %
\fancypagestyle{plain}{
\fancyhf{}
\lhead{}
\chead{\songti \zihao{5} 北京理工大学硕士学位论文}
\rhead{}
\lfoot{}
\cfoot{\songti \zihao{5} \thepage}
\rfoot{}
}

%%%%%%%%%%%%%%%%%%%%%%%%%%%%%%%%%%%%%%%%%%%%%%%%%%%%%%%%%%%%%%%%%%%%%%
% 字体和行距 % 
%%%%%%%%%%%%%%%%%%%%%%%%%%%%%%%%%%%%%%%%%%%%%%%%%%%%%%%%%%%%%%%%%%%%%%
\zihao{-4}
\songti
\baselineskip=22pt

%%%%%%%%%%%%%%%%%%%%%%%%%%%%%%%%%%%%%%%%%%%%%%%%%%%%%%%%%%%%%%%%%%%%%%
% 摘要和目录等 % 
%%%%%%%%%%%%%%%%%%%%%%%%%%%%%%%%%%%%%%%%%%%%%%%%%%%%%%%%%%%%%%%%%%%%%%
\frontmatter
\pagenumbering{Roman}% 罗马数字标记页码 %

\include{chapters/abstract}% 中英文摘要 %

\tableofcontents% 生成目录 %

%\listoftables% 生成表格索引 %
%\addcontentsline{toc}{chapter}{\listtablename}% 将表格索引加入全文目录 %

\listoffigures% 生成插图索引 %
%\addcontentsline{toc}{chapter}{\listfigurename}% 将图索引加入全文目录 %

%%%%%%%%%%%%%%%%%%%%%%%%%%%%%%%%%%%%%%%%%%%%%%%%%%%%%%%%%%%%%%%%%%%%%%
% 正文:各章节内容 %
%%%%%%%%%%%%%%%%%%%%%%%%%%%%%%%%%%%%%%%%%%%%%%%%%%%%%%%%%%%%%%%%%%%%%%
\mainmatter

\include{chapters/01Intro}
\include{chapters/02TeleGra}
\include{chapters/03PNA}
\include{chapters/04ftMassiveGravity}
\include{chapters/05Conclusions}
\include{chapters/appendix} 

%%%%%%%%%%%%%%%%%%%%%%%%%%%%%%%%%%%%%%%%%%%%%%%%%%%%%%%%%%%%%%%%%%%%%%
% 参考文献:请使用BibTex来生成,否则写bibitem将是及痛苦和麻烦的事情。
% 学校的规范如下:
% ” 按照国家有关标准GB/T 7714—2005《文后参考文献著录规则》进行著录。
% 	对该项国家标准未列入的,但又为论文引用的其他类型文献,
% 	可参照社会通用的格式著录。“
%%%%%%%%%%%%%%%%%%%%%%%%%%%%%%%%%%%%%%%%%%%%%%%%%%%%%%%%%%%%%%%%%%%%%%
\bibliographystyle{reference/GBT7714-2005NLang}
{\zihao{5}
\bibliography{reference/ref,reference/zotero}
\addcontentsline{toc}{chapter}{参考文献}
}

%%%%%%%%%%%%%%%%%%%%%%%%%%%%%%%%%%%%%%%%%%%%%%%%%%%%%%%%%%%%%%%%%%%%%%
% 附加资料 %
%%%%%%%%%%%%%%%%%%%%%%%%%%%%%%%%%%%%%%%%%%%%%%%%%%%%%%%%%%%%%%%%%%%%%%
\backmatter
\include{chapters/paper_list}% 攻读学位期间发表论文与研究成果清单 %
\include{chapters/thanks}% 致谢 %

\end{document}
